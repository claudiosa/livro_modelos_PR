
\chapter*{Nota desta Versão Experimental}


\section*{Autores}

Este é um livro experimental (projeto de uma publicação a longo prazo) e tem muitos autores. Dentre eles cita-se:

\begin{enumerate}
\item Marcos Creuz Filho
\item Lucas Hermman Negri
\item ..........
\item Ajudando ... reservado para incluirmos o seu nome  ...

\end{enumerate}


\section*{Sobre o Livro}

\begin{enumerate}
\item Este livro tem um foco: é o título do livro. Alguns modelos
clássicos de problemas, alguns originais, 
são discutidos, modelados e implementados em Minizinc.

\item O Minizinc foi escolhido pela sua leitura próxima a formulação
matemática. Voce irá gostar desta linguagem orientada a modelos;


\item Este livro, parte dele, é resultado de alguns semestres de ensino de graduação no curso de Ciência da Computaçao da UDESC, de uma disciplina optativa introdutória
a Programação por Restrições (PR).

%\item 

\end{enumerate}


\section*{Resumindo ....}

A proposta deste livro é completamente {\bf \textcolor{red}{embrionária}}.  Neste primeiro
momento, há um {\em copy-paste} de um TCC aqui da UDESC--Joinville, que orientei. Contudo, há  muito material escrito
que falta organizar dentro desta proposta.

Assim, vá me cobrando por email, \url{claudio.sa@udesc.br},
que vou aprontando o material na medida que as perguntas surgerirem.
